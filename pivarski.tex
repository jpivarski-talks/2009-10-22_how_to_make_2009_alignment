\documentclass[compress]{beamer}
\usepackage{ifthen,verbatim}

\newcommand{\isnote}{}
\xdefinecolor{lightyellow}{rgb}{1.,1.,0.25}
\xdefinecolor{darkblue}{rgb}{0.1,0.1,0.7}

%% Uncomment this to get annotations
%% \def\notes{\addtocounter{page}{-1}
%%            \renewcommand{\isnote}{*}
%% 	   \beamertemplateshadingbackground{lightyellow}{white}
%%            \begin{frame}
%%            \frametitle{Notes for the previous page (page \insertpagenumber)}
%%            \itemize}
%% \def\endnotes{\enditemize
%% 	      \end{frame}
%%               \beamertemplateshadingbackground{white}{white}
%%               \renewcommand{\isnote}{}}

%% Uncomment this to not get annotations
\def\notes{\comment}
\def\endnotes{\endcomment}

\setbeamertemplate{navigation symbols}{}
\setbeamertemplate{headline}{\mbox{ } \hfill
\begin{minipage}{5.5 cm}
\vspace{-0.75 cm} \small
\end{minipage} \hfill
\begin{minipage}{4.5 cm}
\vspace{-0.75 cm} \small
\begin{flushright}
\ifthenelse{\equal{\insertpagenumber}{1}}{}{Jim Pivarski \hspace{0.2 cm} \insertpagenumber\isnote/\pageref{numpages}}
\end{flushright}
\end{minipage}\mbox{\hspace{0.2 cm}}\includegraphics[height=1 cm]{../cmslogo} \hspace{0.1 cm} \includegraphics[height=1 cm]{../tamulogo} \hspace{0.01 cm} \vspace{-1.05 cm}}

\begin{document}
\begin{frame}
\vfill
\begin{center}
\textcolor{darkblue}{\Large Proposed Method of Producing \\ \vspace{0.2 cm} a CRAFT-2009 Alignment}

\vfill
\begin{columns}
\column{0.3\linewidth}
\begin{center}
\large
\textcolor{darkblue}{Jim Pivarski}
\end{center}
\end{columns}

\begin{columns}
\column{0.3\linewidth}
\begin{center}
\scriptsize
{\it Texas A\&M University}
\end{center}
\end{columns}

\vfill
23 October, 2009

\end{center}
\end{frame}

%% \begin{notes}
%% \item This is the annotated version of my talk.
%% \item If you want the version that I am presenting, download the one
%% labeled ``slides'' on Indico (or just ignore these yellow pages).
%% \item The annotated version is provided for extra detail and a written
%% record of comments that I intend to make orally.
%% \item Yellow notes refer to the content on the {\it previous} page.
%% \item All other slides are identical for the two versions.
%% \end{notes}

\small

\begin{frame}
\frametitle{DT alignment}

\begin{itemize}
\item Here's what I propose for combined DT and CSC alignments, with a
discussion of which parameters are well-determined and which are
poorly-determined with tracks.

\item Proposed DT construction:
\begin{enumerate}
\item 2009 photogrammetry and internal layer alignments from Luca
\item global adjustment of the whole barrel from tracks (in the transverse plane)
\item track-based chamber alignments for accessible chambers, except $\phi_x$ corrections.
\end{enumerate}

\item Where each parameter comes from:
\begin{itemize}
\item all layers/superlayer within chambers: internal alignment
\item all $\phi_x$: photogrammetry
\item $y$, $z$, and $\phi_x$ of station 4 chambers: photogrammetry
\item low-statistics chambers: relative positions from photogrammetry
with an overall global correction from tracks (barrel adjustment)
\item high-statistics chambers: tracks
\end{itemize}

\end{itemize}
\end{frame}

\begin{frame}
\frametitle{DT alignment (continued)}

\begin{itemize}
\item What track-based alignment determines well/poorly:
\begin{itemize}
\item well: relative displacements between tracker and DT layers (as
rigid groups within chambers)
\item well: local $x$, $\phi_y$, $\phi_z$, medium: $y$, $z$, poorly: $\phi_x$
\item well: central chambers above and below tracker, poorly: horizontal
chambers and chambers near the wheel $\pm$2, station 1 corner of the
barrel
\end{itemize}

\end{itemize}
\end{frame}

\begin{frame}
\frametitle{CSC alignment}

\begin{itemize}
\item Proposed CSC construction:
\begin{enumerate}
\item 2007 photogrammetry-derived chambers relative to disks (Karoly)
\item 2009 link-derived chambers relative to disks or rings (Celso)
\item 2009 $\phi_x$ and $z$ from endcap hardware (Jim B.)
\item global adjustment of whole rings from tracks (in the transverse plane)
\item track-based chamber alignments ($x$, $\phi_y$, $\phi_z$) for the few chambers
that have enough statistics.
\end{enumerate}

\item Where each parameter comes from:
\begin{itemize}
\item chamber positions: track-based unless not available due to
statistics, otherwise link-derived (with global adjustment) unless not
available, otherwise photogrammetry (with global adjustment)
\item $z$ of disks, average $\phi_x$ of chambers: endcap hardware
\end{itemize}

\end{itemize}
\end{frame}

\begin{frame}
\frametitle{CSC alignment (continued)}

\begin{itemize}
\item What the track-based alignment determines well/poorly:
\begin{itemize}
\item well: relative displacements between tracker and CSC layers (as
rigid groups within rings)
\item well: local $x$, $\phi_y$, $\phi_z$, poorly: $y$, $z$, $\phi_x$
\item well: chambers on the top and bottom of rings, especially
large-radius rings like ring-2 and ME1/3.
\end{itemize}

\end{itemize}
\end{frame}

\begin{frame}
\frametitle{Last steps}

\begin{itemize}
\item Changes with respect to the presented alignment:
\begin{itemize}
\item full 1st reprocessing dataset: the one I used for the study already
shown was a test-sample of the reprocessing--- that's why the statistics
was a factor of 10 lower than the full sample.
\item any tracker geometry updates (I'm pretty sure an update is planned)
\item new input from endcap hardware and link
\end{itemize}

\item Update: data sample is now available; waiting on the other items

\item Tuesday's AlCa meeting:
\begin{itemize}
\item conflicts with CSC-DPG; I will be busy presenting an endcap update there.
\item will someone else be available to present at AlCa?
\item and ask about status of tracker alignment?
\end{itemize}
\end{itemize}

\label{numpages}
\end{frame}

%% \section*{First section}
%% \begin{frame}
%% \begin{center}
%% \Huge \textcolor{blue}{First section}
%% \end{center}
%% \end{frame}

\end{document}
